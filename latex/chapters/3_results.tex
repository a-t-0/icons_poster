%\section{Results \& Discussion}\label{sec:results}
%\textcolor{red}{TODO: change dummy results with actual results.}
%\textcolor{red}{TODO: Include synaptic death.}
%The domains specified in \cref{subsec:results_energy_consumption} to \cref{subsec:results_radiation_robustness} are evaluated as part of the presentation of the results of this experiment. The first three are used to put the radiation robustness in perspective. All four domains are compared to the reference baseline presented in \cref{sec:methodology}.
\subsection*{Radiation Robustness}\label{subsec:algorithm_performance}

\begin{table}[H]
\caption{Average fraction of correct outputs of the SNN implementation with and without adaptation. Neuron deaths can occur in default and redundant LIF neurons.}
\begin{tabular}{llllll}
        & \multicolumn{5}{l}{Neuron Death Percentage in SNN} \\ \cmidrule{2-6}
        Adaptation & 0\%   5\% & 10\%    & 25\%     \\ \hline
        Without      & 1 &  1    & 0                 \\
        With      & 1 & 1    & 0.5                   \\    
\end{tabular}
\end{table}

%If  multiple brain adaptation mechanisms are implemented before the deadline, use line type to distinguish between these implementations, and generate 3 plots, or use line colour to distinguish between redundancy levels. (Depending on what is most insightful).

\subsection*{Neuronal, Synaptic \& Energy Costs}\label{subsec:results_neuronal_synaptic_overcapacity}
%By taking this factor into account, a context can be provided for the added value of using brain-inspired radiation robustness implementations in neuromorphic hardware.

\begin{table}[H]
\caption{The multiple of the amount of neurons and synapses used by adaptation implementation, compared to the default SNN implementation.}\label{tab:overcapacity}
\begin{tabular}{llll}
           & \multicolumn{3}{l}{Average Cost of Adaptation Mechanism} \\ \cmidrule{2-4}
           Graph Size & Neuronal     & Synaptic & Spikes                   \\ \hline 
3       & 5.7          & 8.3        & 32                \\
4      & 5.8            & 10        & 37.5                  \\
5      & 5.8            & 10.2           & 64              
\end{tabular}
\end{table}\vspace{-1em}
Note this is the average overcapacity for $m=0$ and $m=1$ combined.
%\subsection*{}\label{subsec:results_energy_consumption}
%\begin{table}[H]
%\caption{Fraction of additional spikes consumed by adaptation mechanism.}
%\begin{tabular}{llllll}
%           & \multicolumn{5}{l}{Neuron Death Percentage in SNN.} \\ %\cmidrule{2-6} % Neuron Death in SNN: with/ without adaption%
%          Graph Size & 0\%    & 10\%    & 25\%    & 50\% \\ \hline
%3      & 32    & 4     & 4       & 4             \\
%4      & 37.5      & 4       & 4       & 4            \\
%5      & 4      & 4       & 4       & 4          
%\end{tabular}
%\end{table}
%Note this is the average overcapacity for $m=0$ and $m=1$ combined.
%\textit{Depending on the radiation test method, the energy consumption may be monitored hardwarematically, or estimated softwarematically. This may be challenging in the case of reliance on neuromorphic cloud architectures for testing, as some may not provide the ability to measure and/or report energy consumption. An approximation to the energy consumption could be counting the occurrence of neuron spikes. This is because neuromorphic architectures are considered to only consume significant amounts of energy when they spike.}

%\subsection*{Time Complexity}\label{subsec:time_complexity}
%TODO:Presents the time complexity of the default SNN as two terms: network initialisation and network runtime. Then presents the same parameters for the adapted SNN. If relevant, as a function of the level of redundancy.

%\subsection*{Space Complexity}\label{subsec:space_complexity}
%TODO:Presents the space complexity of the default SNN as two terms: network initialisation and network runtime. Then presents the same parameters for the adapted SNN. If relevant, as a function of the level of redundancy.
