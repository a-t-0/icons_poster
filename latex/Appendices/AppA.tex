\section{Appendix}\label{app:1}
For transparency the source code for the generated Gantt chart is included. To recreate the Gantt chart, first install the package diagrams in a python 3.6,3.7 or 3.8 environment. One can do this in Anaconda prompt by opening it and entering:
\begin{verbatim}
conda update -n base -c defaults conda
conda create -n py36 python=3.6
conda activate py36
pip install diagrams
pip install plantuml
\end{verbatim}

\noindent An individual Gantt chart diagram can be generated by browsing in anaconda to the direction of the \verb+<diagram name>.txt+ file, and processing it with command \verb+python -m plantuml <diagram name>.txt+. However, one can also automatically (re-)generate all the diagrams encoded in the \verb+.txt+ files in the \verb+../../code/project3/Diagrams/+ folder by opening Anaconda prompt, browsing to the directory of \verb+../../code/project3/+ and running: \verb+python create_diagrams.py+. The code of \verb+python create_diagrams.py+ is included in \cref{app:1_subsec_py}, and to enable people with visual impairments to read the diagrams, the source codes of the diagrams are included in \cref{app:1_subsec_flow_diagram_sources}.

\noindent Additional example graphs and graph generation syntax is specified in: \url{https://plantuml.com/wbs-diagram}.

\newpage
%\section{Code used to generate diagrams}\label{app:1_subsec_py}
%\pythonexternal{../../code/project6/create_diagrams_locally.py}

\newpage
\section{Source Code of Flow Diagrams}\label{app:1_sec_flow_diagram_sources}
The code of the Gantt chart that contains the project planning, is listed as:
%\VerbatimInput[label=\fbox{\color{Black}Gannt Chart.txt}]{../../code/5_Project_plan/Diagrams/updated_gantt.uml}

\VerbatimInput[label=\fbox{gantt.uml}]{code/project6/src/Diagrams/gantt.uml}

\newpage
\section{Source Code of Organogram}\label{app:1_sec_organogram}
The code of the organogram presented in \cref{sec:organogram}, is listed as:
\VerbatimInput[label=\fbox{organogram.uml}]{code/project6/src/Static_diagrams/organogram.uml}